%% DON'T DO THIS IN ACM STYLE
%%\renewcommand{\baselinestretch}{0.964}

\newcommand{\tname}{\textsc{Shaker}} %% technique name
\newcommand{\rerun}{ReRun}
\newcommand{\rerunN}{ReRun$_N$}
\newcommand{\ngen}{$\mathit{NG}$}
\newcommand{\sng}{stress-ng}
\newcommand{\artifactUrl}{\url{https://github.com/shaker-project/shaker}}
\newcommand{\greedy}{Greedy}

\newcommand{\tr}{$\mathit{TR}$}

\newcommand{\Mar}[1]{[\textbf{Marcelo}:{\color{magenta} #1}]}
\newcommand{\Leo}[1]{[\textbf{Leopoldo}:~{\color{blue} #1}]}
\newcommand{\Den}[1]{[\textbf{Denini}:~{\color{brown} #1}]}

\newcommand{\CodeIn}[1]{\begin{footnotesize}\texttt{#1}\end{footnotesize}}
\newcommand{\Fix}[1]{\textbf{[\color{red}#1]}}
\newcommand{\Ignore}[1]{}
\newcommand{\myeg}{e.g.}
\newcommand{\ie}{i.e.}
\newcommand{\etc}{etc.}

\newcommand{\ann}{ANN}
\newcommand{\bmc}{BMC}
\newcommand{\ibmc}{IBMC}
\newcommand{\etal}{et al.}
\renewcommand{\algorithmicrequire}{\textbf{Input:}}
\renewcommand{\algorithmicensure}{\textbf{Output:}}
\newtheorem{theorem}{Theorem}[section]
\newtheorem{corollary}{Corollary}[theorem]
\newtheorem{lemma}[theorem]{Lemma}

%\lstset{language=C,basicstyle=\small}
\lstset{language=C,basicstyle=\small\ttfamily}
\lstset{numbers=left, numberstyle=\tiny, stepnumber=1, numbersep=5pt}
\lstset{tabsize=2}
\lstset{firstnumber=1}
\lstset{frame=single}
%\lstset{float}
\lstset{
  basicstyle=\scriptsize\ttfamily,
  keywordstyle=\scriptsize\ttfamily\bfseries,
  language=C,             % choose the language of the code
  %frame=, %single             % adds a frame around the code
  aboveskip=0pt,
  belowskip=0pt,
  breaklines=true,           % sets automatic line breaking
  breakatwhitespace=false,   % sets if automatic breaks should only happen at
  showspaces=false,
  %tabsize=2,                  % Groesse von Tabs
  %extendedchars=true,         %
  %breaklines=true,            % Zeilen werden Umgebrochen
  %keywords=[2]{},
  keywords={},
  %% numbersep=0pt,              % Abstand der Nummern zum Text  
  %% numbers=left,
  escapeinside={\%*}{*)},          % if you want to add LaTeX within  % your code
  morekeywords={public, for, typedef, void, float, unsigned, short, int, ushort, assert,uchar,begin_thread,end_thread,join_thread,atomic,assume,static,extern,int,_Bool,return}  
}

\newcommand{\Space}[1]{}

% other paragraphs
\newcommand{\Contrib}[1]{$\star$#1}


%% numbers 
\newcommand{\numprojects}{11}
\newcommand{\totalTests}{1,298}
\newcommand{\numprojectsWithFlakies}{7}
\newcommand{\numNewFlakies}{30}
\newcommand{\numReRuns}{50}
\newcommand{\numflakytestsds}{75}
\newcommand{\percFlakyTestSds}{5.78}
\newcommand{\numflakytraining}{35}
\newcommand{\numflakytesting}{40}
\newcommand{\numReRunstraining}{3}
\newcommand{\numReRunstrainingLONG}{3}
\newcommand{\numTestsAntenna}{250}
\newcommand{\numTrainingSet}{35}
\newcommand{\numTestingSet}{40}
\newcommand{\numRandomConfigs}{50}
\newcommand{\threshold}{0.66}
\newcommand{\numExecutionsRQThree}{10}
\newcommand{\numTotalNewFlaky}{61}
%RQ4
\newcommand{\numFlakyDetectedShaker}{38}
\newcommand{\numFlakyDetectedReRun}{15}
\newcommand{\percFlakyDetectedShaker}{95\%}
\newcommand{\percFlakyDetectedReRun}{37.5\%}


%%numbersRQ3
\newcommand{\configsMHS}{$\{c_1, c_18, c_20, c_17\}$}
\newcommand{\configsGreed}{$\{c_18, c_20, c_22, c_32\}$}
\newcommand{\configsRandom}{$\{c_1,...\}$}
